\section{Introducción}

Iniciamos este trabajo con el objetivo de aprender sobre el protocolo ARP y, más en general, las redes de Internet y sus algoritmos de intercambio de paquetes. 

ARP es un protocolo de resolución de direcciones, mapea direcciones $IP$ a direcciones $MAC$ (es decir que va del nivel de red al nivel de enlace). Se compone de dos tipos de mensajes: \emph{who-has} e \emph{is-at}

A lo largo de este informe aplicaremos el conocimiento adquirido sobre teoría de la información para analizar propiedades en la red y detectar símbolos destacados. A partir de esta metodología creemos que podremos entre otras cosas diferenciar routers de hosts.

Nuestra hipótesis es que los routers destacarán en el intercambio de paquetes who-has siendo los que mas aparezcan como destinatarios del mismo.
