
Como resultado final pudimos identificar de distintas maneras nodos destacados, por ejemplo a partir del grafo subyacente que armamos pudimos identificar en todos los casos al gateway a partir del nodo de mayor grado. A continuación describimos brevemente cada caso analizado por separado.

\subsubsection{Red Domiciliaria}

En la red domiciliaria se pudo ver en el grafo resultante como destacaban dos nodos. Uno de ellos, el de mayor grado, era el gateway, el otro creemos que fue la computadora que tomó las mediciones ya que estaba accediendo a multiples sitios de internet lo cual provoco un gran intercambio de paquetes.

\subsubsection{Starbucks}

Quizas por una mala configuración de la red o por la poca clientela que había en este local en el horario en que se tomaron las mediciones (7:10 AM) es que este grafo resulto ser tan pobre, de todas maneras se pueden observar perfectamente los dos nodos que uno esperaría ver como minimo. El gateway y la computadora que tomo las mediciones.

\subsubsection{Laboratorio de Computación}

Este es el grafo mas grande y en el cual se vuelve aún mas claro el patrón que veniamos viendo, el nodo 10.2.203.254 resalta completamente de los demás y concuerda perfectamente con nuestra hipotesis de que el gateway es el que mas paquetes intercambia en este protocolo, lo cual lo vuelve un nodo destacado.
